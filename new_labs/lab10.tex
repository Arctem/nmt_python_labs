% LAB 10: Markov Chains
% 
% CSE/IT 107: Introduction to Programming
% New Mexico Tech
% 
% Prepared by Russell White and Christopher Koch
% Fall 2014
\documentclass[11pt]{cselabheader}

%%%%%%%%%%%%%%%%%% SET TITLES %%%%%%%%%%%%%%%%%%%%%%%%%
\fancyhead[R]{Lab 10: Markov Chains}
\title{Lab 10: Markov Chains}

\begin{document}

\maketitle

\hrule
\begin{quotation}
``Holy shit, you geeks are badass.''
\end{quotation}
\begin{flushright}
  --- Pam (\emph{Archer})
\end{flushright}

\begin{quotation}
``Simplicity is prerequisite for reliability.''
\end{quotation}
\begin{flushright}
--- Edsger W. Dijkstra
\end{flushright}

\begin{quotation}
``Simplicity is the final achievement. After one has played a vast quantity of
notes and more notes, it is simplicity that emerges as the crowning reward of
art.''
\end{quotation}
\begin{flushright}
--- Fr\`ed\`eric Chopin
\end{flushright}

\begin{quotation}
``The truth is a trap: you can not get it without it getting you; you cannot get
the truth by capturing it, only by its capturing you.''
\end{quotation}
\begin{flushright}
--- S{\o}ren Kierkegaard
\end{flushright}

\hrule

\section{Introduction}

\pagebreak
\section{Exercises}
\label{sec:ex}

\begin{warningbox}{Boilerplate}
  Remember that this lab \emph{must} use the
  boilerplate syntax introduced in Lab~5.
\end{warningbox}

\section{Submitting}

Files to submit:
\begin{itemize}
\item
\end{itemize}

You may submit your code as either a tarball (instructions below) or as a .zip
file. Either one should contain all files used in the exercises for this lab.
The submitted file should be named either
\texttt{cse107\_firstname\_lastname\_lab10.zip} or
\texttt{cse107\_firstname\_lastname\_lab10.tar.gz} depending on which method you
used.

For Windows, use a tool you like to create a \texttt{.zip} file. The TCC
computers should have \texttt{7z} installed. For Linux, look at lab 1 for
instructions on how to create a tarball or use the ``Archive Manager'' graphical
tool.

\begin{center}
  \textbf{Upload your tarball or .zip file to Canvas.}
\end{center}

\end{document}
