% LAB 5: File I/O
% 
% CSE/IT 107: Introduction to Programming
% New Mexico Tech
% 
% Prepared by Russell White and Christopher Koch
% Fall 2014
\documentclass[11pt]{cselabheader}

%%%%%%%%%%%%%%%%%% SET TITLES %%%%%%%%%%%%%%%%%%%%%%%%%
\fancyhead[R]{Lab 5: File I/O}
\title{Lab 5: File I/O}

\begin{document}

\maketitle

\hrule
\begin{quotation}
  ``If you don't think carefully, you might believe that programming is just
  typing statements in a programming language.''
\end{quotation}
\begin{flushright}
  --- W. Cunningham
\end{flushright}

\begin{quotation}
  ``Only ugly languages become popular. Python is the exception.''
\end{quotation}
\begin{flushright}
  --- Donald Knuth
\end{flushright}

\hrule

\section{Introduction}


\section{File I/O}


\section{Program Boilerplate}


%\section{Recursion}


\section{Exercises}
\label{sec:ex}

\begin{description}
  \item[something.py] test
\end{description}

\section{Submitting}

Files to submit:
\begin{itemize}
  \item hailstone.py
\end{itemize}

You may submit your code as either a tarball (instructions below) or as a .zip
file. Either one should contain all files used in the exercises for this lab.
The submitted file should be named either
\texttt{cse107\_firstname\_lastname\_lab5.zip} or
\texttt{cse107\_firstname\_lastname\_lab5.tar.gz} depending on which method you
used.

For Windows, use a tool you like to create a \texttt{.zip} file. The TCC
computers should have \texttt{7z} installed. For Linux, look at lab 1 for
instructions on how to create a tarball or use the ``Archive Manager'' graphical
tool.

\begin{center}
  \textbf{Upload your tarball or .zip file to Canvas.}
\end{center}

\end{document}
