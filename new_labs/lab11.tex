% LAB 11: Web Scraping
%
% CSE/IT 107: Introduction to Programming
% New Mexico Tech
%
% Prepared by Russell White and Christopher Koch and Tyler Cecil
% Fall 2014
\documentclass[11pt]{cselabheader}

%%%%%%%%%%%%%%%%%% SET TITLES %%%%%%%%%%%%%%%%%%%%%%%%%
\fancyhead[R]{Lab 11: Web Scraping}
\title{Lab 11: Web Scraping}

\begin{document}

\maketitle

\hrule
\begin{quotation}
``The sooner you start to code, the longer the program will take.''
\end{quotation}
\begin{flushright}
--- R. Carlson
\end{flushright}

\begin{quotation}
``Any sufficiently advanced technology is indistinguishable from magic.''
\end{quotation}
\begin{flushright}
--- A. Clarke
\end{flushright}

\begin{quotation}
``Imagination is more important than knowledge.''
\end{quotation}
\begin{flushright}
--- A. Einstein
\end{flushright}

\hrule

\section{Introduction}

\subsection{HTML}

As you may be familiar, HTML is the main formatting language of the
Internet.

\subsection{BeautifulSoup}

What do?

How do?

We gave it to you.

Example

\subsection{Matplotlib}

What do?

How do?

\subsection{Wikipedia}

That one thing that russell found (about the philosophy link)

\pagebreak
\section{Exercises}
\label{sec:ex}

%One exerscis for soup (like get the title of a page)
%One for matplotlib (plot x^2)
%The main project

\begin{warningbox}{Boilerplate}
  Remember that this lab \emph{must} use the
  boilerplate syntax introduced in Lab~5.
\end{warningbox}

\begin{description}
  \item[file.py]
\end{description}

\pagebreak
\section{Submitting}

Files to submit:
\begin{itemize}
\item file.py (Section~\ref{sec:ex})
\end{itemize}

You may submit your code as either a tarball (instructions below) or as a .zip
file. Either one should contain all files used in the exercises for this lab.
The submitted file should be named either
\texttt{cse107\_firstname\_lastname\_lab11.zip} or
\texttt{cse107\_firstname\_lastname\_lab11.tar.gz} depending on which method you
used.

For Windows, use a tool you like to create a \texttt{.zip} file. The TCC
computers should have \texttt{7z} installed. For Linux, look at lab 1 for
instructions on how to create a tarball or use the ``Archive Manager'' graphical
tool.

\begin{center}
  \textbf{Upload your tarball or .zip file to Canvas.}
\end{center}

\end{document}
