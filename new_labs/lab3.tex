% LAB 3: Lists and Strings
% 
% CSE/IT 107: Introduction to Programming
% New Mexico Tech
% 
% Prepared by Russell White and Christopher Koch
% Fall 2014
\documentclass[11pt]{cselabheader}

%%%%%%%%%%%%%%%%%% SET TITLES %%%%%%%%%%%%%%%%%%%%%%%%%
\fancyhead[R]{Lab 3: Lists and Strings}
\title{Lab 2: Basic Flow Control}

\begin{document}

\maketitle

\hrule

\begin{quotation}
``When you come to a fork in the road, take it."
\end{quotation}
\begin{flushright}
--- Attributed to Yogi Berra
\end{flushright}

\begin{quotation}
``Simplicity is the ultimate sophistication.''
\end{quotation}
\begin{flushright}
--- Leonardo Da Vinci
\end{flushright}

\begin{quotation}
``How do we convince people that in programming simplicity and clarity -- in
short: what mathematicians call elegance -- are not a dispensable luxury, but
a crucial matter that decides between success and failure?''
\end{quotation}
\begin{flushright}
--- Edsger Dijkstra
\end{flushright}

\hrule

\section{Introduction}


\pagebreak
\section{Submitting}

Files to submit:
\begin{itemize}
%  \item conversions.py (see Section~\ref{subsec:ifex})
\end{itemize}

You may submit your code as either a tarball (instructions below) or as a .zip
file. Either one should contain all files used in the exercises for this lab.
The submitted file should be named either
\texttt{cse107\_firstname\_lastname\_lab3.zip} or
\texttt{cse107\_firstname\_lastname\_lab3.tar.gz} depending on which method you
used.

For Windows, use a tool you like to create a \texttt{.zip} file. The TCC computers should
have \texttt{7z} installed. For Linux, look at lab 1 for instructions on how to
create a tarball or use the ``Archive Manager'' graphical tool.

\begin{center}
  \textbf{Upload your tarball or .zip file to Canvas.}
\end{center}

\end{document}
