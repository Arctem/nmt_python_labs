% LAB 3: Lists and Strings
% 
% CSE/IT 107: Introduction to Programming
% New Mexico Tech
% 
% Prepared by Russell White and Christopher Koch
% Fall 2014
\documentclass[11pt]{cselabheader}

%%%%%%%%%%%%%%%%%% SET TITLES %%%%%%%%%%%%%%%%%%%%%%%%%
\fancyhead[R]{Lab 3: Lists and Strings}
\title{Lab 3: Lists and Strings}

\begin{document}

\maketitle

\hrule

\begin{quotation}
``Each decision we make, each action we take, is born out of an intention."
\end{quotation}
\begin{flushright}
--- Sharon Salzberg
\end{flushright}

\begin{quotation}
``Programming is learned by writing programs.''
\end{quotation}
\begin{flushright}
--- Brian Kernighan
\end{flushright}

\begin{quotation}
``The purpose of computing is insight, not numbers.''
\end{quotation}
\begin{flushright}
--- Richard Hamming, 1962
\end{flushright}

\hrule

\section{Introduction}
In the first two labs, we showed you how to generally use Python and how to
have your programs make decisions based on what the user entered. With this
lab, we will be starting to show you how to do some useful things in Python:
how to make lists and manipulate them and how to deal with strings. You have
seen strings before, but we will be showing you how to do neat things to them.

\section{Lists}
\subsection{Summary}

\subsection{Exercises}
\label{subsec:listsex}

\section{Strings}

You have seen strings in Python before. They are sequences of characters
enclosed by either double quotes or single quotes; for example:
\begin{lstlisting}[style=ipython,language=Octave] % string escaping issues with Python style
>>> s = "I'm a string."
>>> print(s)
I'm a string.
>>> r = 'I am also a string.'
>>> print(r)
I'm also a string.
\end{lstlisting}

Notice how we did not use a single quote in the second string because it was
enclosed (\emph{delimited}) by single quotes. The proper way to use a single
quote in a single quoted string or a double quote in a double quoted string
goes like this:
\begin{lstlisting}[style=ipython,language=Octave]
>>> s = "Previously, we said \"I'm a string.\"."
>>> print(s)
Previously, we said "I'm a string.".
>>> r = 'I\'m also a string.'
>>> print(r)
I'm also a string.
\end{lstlisting}
This is called \emph{escaping} a character. We \emph{escaped} the double quotes
and single quote respectively so that Python did not think it was the end of
the string.

% show ' in "
% show " in '
% show \n in '
% show raw for \n

We also saw indirectly and previously that we can concatenate strings together
using the addition operator \lstinline!+!:
\begin{lstlisting}[style=ipython,language=Octave] % octave doesn't have in
keyword
>>> s = "The cat"
>>> r = " in the hat"
>>> t = s+r
>>> print(t)
The cat in the hat
\end{lstlisting}

In addition to that, we can repeat strings using the multiplication operator
\lstinline!*!: 
\begin{lstlisting}[style=ipython]
>>> s = "Hi"
>>> r = 5*s
>>> print(r)
HiHiHiHiHi
\end{lstlisting}

% literals next to each other concat
% strings immutable
% create new string if need change
% len()

You previously saw \emph{slicing} in the section on lists. Slicing works on strings, too!
\begin{lstlisting}[style=ipython]
>>> s = "The cat in the hat"
>>> print(s[4:7])
cat
>>> print(s[15:18])
hat
>>> print(s[14:18])
 hat
>>> print(s[17:14:-1])
tah
>>> print(s[17::-1])
tah eht ni tac ehT
\end{lstlisting}


If you want strings to go on for multiple lines, you have to use three double
quotes:
\begin{lstlisting}
weizsaecker = """We in the older generation owe to young people not the 
fulfillment of dreams but honesty. We must help younger people to 
understand why it is vital to keep memories alive. We want to help them 
to accept historical truth soberly, not one-sidedly, without taking 
refuge in utopian doctrines, but also without moral arrogance. From our
own history we learn what man is capable of. For that reason we must not
imagine that we are quite different and have become better. There is no
ultimately achievable moral perfection. We have learned as human beings,
and as human beings we remain in danger. But we have the strength to 
overcome such danger again and again."""
\end{lstlisting} 



\subsection{Summary}

\subsection{Exercises}
\label{subsec:stringsex}

\section{For Loops Again}
\subsection{Summary}

\subsection{Exercises}
\label{subsec:forex}


\pagebreak
\section{Submitting}

Files to submit:
\begin{itemize}
  \item .py (see Section~)
\end{itemize}

You may submit your code as either a tarball (instructions below) or as a .zip
file. Either one should contain all files used in the exercises for this lab.
The submitted file should be named either
\texttt{cse107\_firstname\_lastname\_lab3.zip} or
\texttt{cse107\_firstname\_lastname\_lab3.tar.gz} depending on which method you
used.

For Windows, use a tool you like to create a \texttt{.zip} file. The TCC computers should
have \texttt{7z} installed. For Linux, look at lab 1 for instructions on how to
create a tarball or use the ``Archive Manager'' graphical tool.

\begin{center}
  \textbf{Upload your tarball or .zip file to Canvas.}
\end{center}

\end{document}
