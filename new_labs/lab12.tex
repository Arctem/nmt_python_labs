% LAB 12: Project
%
% CSE/IT 107: Introduction to Programming
% New Mexico Tech
%
% Prepared by Russell White and Christopher Koch and Tyler Cecil
% Fall 2014
\documentclass[11pt]{cselabheader}
\usepackage{multicol,graphicx}

%%%%%%%%%%%%%%%%%% SET TITLES %%%%%%%%%%%%%%%%%%%%%%%%%
\fancyhead[R]{Lab 12: Project}
\title{Lab 12: Project}

\begin{document}

\maketitle

\hrule
\begin{quotation}
``The limits of my language mean the limits of my world.''
\end{quotation}
\begin{flushright}
--- Ludwig Wittgenstein
\end{flushright}

\begin{quotation}
``Any sufficiently advanced technology is indistinguishable from magic.''
\end{quotation}
\begin{flushright}
--- Arthur Clarke
\end{flushright}

\begin{quotation}
``Imagination is more important than knowledge.''
\end{quotation}
\begin{flushright}
--- Albert Einstein
\end{flushright}

\hrule

\section{Introduction}
This will be your final lab for the semster.

\pagebreak

\section{tkinter}
\label{sec:tk}

\pagebreak

\section{Exercises}
\label{sec:ex}

\begin{warningbox}{Boilerplate}
  Remember that this lab \emph{must} use the
  boilerplate syntax introduced in Lab~5.
\end{warningbox}

\begin{description}
\item[battle.py]
\item[battle\_server.py]
\end{description}

\pagebreak
\section{Submitting}

Files to submit:
\begin{itemize}
\item battle.py (Section~\ref{sec:ex})
\item battle\_server.py (Section~\ref{sec:ex})
\end{itemize}

You may submit your code as either a tarball (instructions below) or as a .zip
file. Either one should contain all files used in the exercises for this lab.
The submitted file should be named either
\texttt{cse107\_firstname\_lastname\_lab11.zip} or
\texttt{cse107\_firstname\_lastname\_lab11.tar.gz} depending on which method you
used.

For Windows, use a tool you like to create a \texttt{.zip} file. The TCC
computers should have \texttt{7z} installed. For Linux, look at lab 1 for
instructions on how to create a tarball or use the ``Archive Manager'' graphical
tool.

\begin{center}
  \textbf{Upload your tarball or .zip file to Canvas.}
\end{center}

\end{document}
