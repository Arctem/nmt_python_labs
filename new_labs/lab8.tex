% LAB 8: Advanced Functions
% 
% CSE/IT 107: Introduction to Programming
% New Mexico Tech
% 
% Prepared by Russell White and Christopher Koch
% Fall 2014
\documentclass[11pt]{cselabheader}
\usepackage{IEEEtrantools}

%%%%%%%%%%%%%%%%%% SET TITLES %%%%%%%%%%%%%%%%%%%%%%%%%
\fancyhead[R]{Lab 8: Advanced Functions}
\title{Lab 8: Advanced Functions}

\begin{document}

\maketitle

\hrule
\begin{quotation}
``All thought is a kind of computation.''
\end{quotation}
\begin{flushright}
--- D. Hobbes
\end{flushright}

\begin{quotation}
``Vague and nebulous is the beginning of all things, but not their end.''
\end{quotation}
\begin{flushright}
--- K. Gibran
\end{flushright}

\begin{quotation}
``It [programming] is the only job I can think of where I get to be both an
engineer and artist. There's an incredible, rigorous, technical element to it,
which I like because you have to do very precise thinking. On the other hand, it
has a wildly creative side where the boundaries of imagination are the only real
limitation.''
\end{quotation}
\begin{flushright}
--- A. Hertzfeld
\end{flushright}

\hrule

\section{Introduction}


\pagebreak
\section{Advanced Functions}
\label{sec:advfun}

\subsection{Default Arguments}
\label{subsec:arg}


\subsection{Recursion}
\label{subsec:recur}


\subsection{Lambda Functions}
\label{subsec:lambda}


\section{Iterables}
\label{subsec:iter}

\subsection{Map}
\label{subsec:map}


\subsection{Reduce}
\label{subsec:reduce}






\pagebreak


%\pagebreak
\section{Exercises}
\label{sec:ex}

\begin{warningbox}{Boilerplate}
  Remember that this lab \emph{must} use the
  boilerplate syntax introduced in Lab~5, including the review exercises.
\end{warningbox}

\begin{description}
  \item[exercise.py]
\end{description}

\pagebreak
\section{Submitting}

\begin{center}
  \textbf{We will be adding more exercises later. We have just not had the time
  to finish them. You will get an email about them.}
\end{center}

Files to submit:
\begin{itemize}
  \item exercise.py (Section~\ref{sec:ex})
\end{itemize}

You may submit your code as either a tarball (instructions below) or as a .zip
file. Either one should contain all files used in the exercises for this lab.
The submitted file should be named either
\texttt{cse107\_firstname\_lastname\_lab8.zip} or
\texttt{cse107\_firstname\_lastname\_lab8.tar.gz} depending on which method you
used.

For Windows, use a tool you like to create a \texttt{.zip} file. The TCC
computers should have \texttt{7z} installed. For Linux, look at lab 1 for
instructions on how to create a tarball or use the ``Archive Manager'' graphical
tool.

\begin{center}
  \textbf{Upload your tarball or .zip file to Canvas.}
\end{center}

\end{document}
