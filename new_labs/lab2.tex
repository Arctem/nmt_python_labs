% LAB 2: Basic Flow Control
% 
% CSE/IT 107: Introduction to Programming
% New Mexico Tech
% 
% Prepared by Cynthia Veitch, William Kwan, Scott Chadde, Kaley Goatcher,
% Russell White, and Christopher Koch
% Fall 2013

\documentclass[11pt,hidelinks]{article}

% Allows for bold tele-type
\DeclareFontShape{OT1}{cmtt}{bx}{n}{
  <5><6><7><8><9><10><10.95><12><14.4><17.28><20.74><24.88>cmttb10}{}

% AMS math formatting
\usepackage{amsmath}
\usepackage{amssymb}
\usepackage{amsthm}
\usepackage{textcomp}
% Indention formatting
\setlength{\parindent}{0.0in}
\setlength{\parskip}{0.05in}

% Margin support
\usepackage[margin=1in]{geometry}

% Header/footer support
\usepackage{fancyhdr}
\pagestyle{fancy}
\fancyhead{}
\fancyhead[L]{CSE 107}
%%%%%%%%%%%%%%%%%%%%%%%%%%%%%%%%%%%%%%%%%%%%%%%%%%%%%%%%%%%%%%%%%%%%%%%%%%%
% Replace with title for lab
\fancyhead[R]{Lab 2: Basic Flow Control}
%%%%%%%%%%%%%%%%%%%%%%%%%%%%%%%%%%%%%%%%%%%%%%%%%%%%%%%%%%%%%%%%%%%%%%%%%%%

% Color support
\usepackage{color}
\usepackage[dvipsnames]{xcolor}
\definecolor{light-gray}{gray}{0.9}
\definecolor{green}{RGB}{0,127,0}
\definecolor{dred}{rgb}{0.6,0,0}
\definecolor{dkbrown}{RGB}{92,51,23}
% Outline support
%\usepackage{outlines}
% URL formatting
\usepackage{}\usepackage{url}
% Image/graphic support
\usepackage{graphicx}
% Supports in-document hyperlinks
\usepackage[pdfborder=0in,bookmarks=true]{hyperref}
\usepackage[numbered]{bookmark}
% List compression
\usepackage{mdwlist}
\usepackage{enumerate}
% Support long tables across pages
\usepackage{longtable}
\setlength{\LTcapwidth}{6in}
\usepackage{array}
\usepackage{multirow}
% better table formatting
\usepackage{booktabs}
% Supports code formatting/highlighting
\usepackage{listings}

\usepackage{MnSymbol}
\lstdefinestyle{python}{
  language=Python,
  basicstyle=\small\ttfamily,
  frame=single,
  prebreak=\raisebox{0ex}[0ex][0ex]{\ensuremath{\rhookswarrow}},
  postbreak=\raisebox{0ex}[0ex][0ex]{\ensuremath{\rcurvearrowse\space}}
  breaklines=true,
  breakatwhitespace=true,
  numbers=left,
  numberstyle=\scriptsize,
  morekeywords={True,False},
  identifierstyle=\color{black},
}
\lstset{
  style=python,
  basicstyle=\small\ttfamily,
  keywordstyle=\color{blue},
  stringstyle=\color{Maroon},
  commentstyle=\color{green},
  showstringspaces=false,
}
\lstdefinestyle{bash}{
  language=bash,
  basicstyle=\small\ttfamily,
  identifierstyle=,
  keywordstyle=,
  backgroundcolor=\color{light-gray}
}

% I wanted bash to have the breaklines stuff, too, but the typesetting is having
% problems with my prebreak and postbreak symbols and boxes. I have to
% investigate (it always worked in the C labs..) -Chris
\lstdefinestyle{c}{
  language=C,
  basicstyle=\small\ttfamily,
  frame=single,
  prebreak=\raisebox{0ex}[0ex][0ex]{\ensuremath{\rhookswarrow}},
  postbreak=\raisebox{0ex}[0ex][0ex]{\ensuremath{\rcurvearrowse\space}}
  breaklines=true,
  breakatwhitespace=true,
  numbers=left,
  numberstyle=\scriptsize
}

% caption formatting
\usepackage[format=plain,font=small,labelfont=bf]{caption}
\usepackage[toc,page]{appendix}
\usepackage{tikz}
\usetikzlibrary{trees}

%%%%%%%%%%%%%%%%%%%%%%%%%%%%%%%%%%%%%%%%%%%%%%%%%%%%%%%%%%%%%%%%%%%%%%%%%%%
% Replace with title for lab
\title{Lab 2: Basic Flow Control}
%%%%%%%%%%%%%%%%%%%%%%%%%%%%%%%%%%%%%%%%%%%%%%%%%%%%%%%%%%%%%%%%%%%%%%%%%%%
\author{CSE/IT 107}
\date{NMT Computer Science}

\begin{document}

\maketitle

\hrule

\begin{quotation}
``When you come to a fork in the road, take it."
\end{quotation}
\begin{flushright}
--- Attributed to Yogi Berra
\end{flushright}

\begin{quotation}
``Simplicity is the ultimate sophistication.''
\end{quotation}
\begin{flushright}
--- Leonardo Da Vinci
\end{flushright}

\begin{quotation}
``How do we convince people that in programming simplicity and clarity -- in
short: what mathematicians call elegance -- are not a dispensable luxury, but
a crucial matter that decides between success and failure?''
\end{quotation}
\begin{flushright}
--- Edsger Dijkstra
\end{flushright}

\hrule

%%%%%%%%%%%%%%%%%%%%%%%%%%%%%%%%%%%%%%%%%%%%%%%%%%%%%%%%%%%%%%%%%%%%%%%%%%%
%%%%%%%%%%%%%%%%%%%%%%%%%%%%%%%%%%%%%%%%%%%%%%%%%%%%%%%%%%%%%%%%%%%%%%%%%%%
\section{Introduction}
The purpose of this lab is to introduce you to the fundamentals of what
programmers call flow control. In the previous lab, we showed you how to do
basic calculations in Python. For example, we had you convert temperature from
Celsius to Fahrenheit and Kelvin.

What if the user of your conversion program wanted to have only one conversion
and you did not know which? We have to be able to give the user a choice. In the
previous lab, you learned about the \lstinline!input()! function that let you
``ask'' the user of your program a question. In this lab, you will learn how to
use \lstinline!if!, \lstinline!else!, and \lstinline!elif! to have the program
choose one action out of multiple actions; for example, whether to convert to
Kelvin or to Fahrenheit.

Sometimes, you also want to be able to repeat a calculation for different
values. For example, you want to calculate the square root of all numbers
between 1 and 100. To do this, you do not have to actually repeat writing the
calculation in your code, there is the \lstinline!while! statement to help you
repeat code.

\subsection{New Code Coloring in PDFs}
% also function names that are self-defined, but let's not talk about that..
In the new code style, all variable names will be black, all keywords will be
blue, all strings will be maroon (red), while comments are green.

\begin{lstlisting}
x = 5
print(x)
print("string")
# I'm a comment
\end{lstlisting}

\section{Boolean logic}
%Cover basic boolean logic with comparisons and whatnot, then use those to drive if and else statements.
A common activity when programming is determining if something value is true or
false. For example, if a variable is less than five or if the user entered the
correct password. Any statement that can be resolved into a true or a false
value is called a boolean statement, the value it resolves into (true or false)
is called a boolean value.

\begin{lstlisting}
>>> x = 5
>>> print(x < 3)
False
>>> print(x < 6)
True
\end{lstlisting}

In the above example, the boolean values are \lstinline{True} and \lstinline{False}. The boolean statements are \lstinline{x < 3} and \lstinline{x < 6}.

In addition to \lstinline{<}, we can also test for other inequalities.

\begin{lstlisting}
>>> x = 3
>>> y = 6
>>> print(x < y)
True
>>> print(x > y)
False
>>> print(x <= y)
True
>>> print(x >= y)
False
\end{lstlisting}

Note that \lstinline{<=} means ``less than or equal to'' and \lstinline{>=} means ``greater than or equal to''.

Finally, we can test if two values are equal (\lstinline{==}) or not equal (\lstinline{!=}).

\begin{lstlisting}
>>> x = 3
>>> y = 3
>>> z = 4
>>> print(x == y)
True
>>> print(x == z)
False
>>> print(y != 5)
True
>>> print(y != x)
False
\end{lstlisting}

It is important to remember that we use \lstinline{=} to assign a value to a variable and \lstinline{==} to test if two values are equal.

\begin{table}
  \centering
  \begin{tabular}{ll}
    \toprule
    Operator & What it tests\\
    \midrule
    \lstinline!a < b! & is $a$ less than $b$ \\
    \lstinline!a > b! & is $a$ greater than $b$ \\
    \lstinline!a <= b! & is $a$ less than or equal to $b$ \\
    \lstinline!a >= b! & is $a$ greater than or equal to $b$ \\
    \lstinline!a == b! & is $a$ equal to $b$ \\
    % because lstinline is retarded and doesn't work right in tables.
    \lstinline!a! !\lstinline!= b! & is $a$ not equal to $b$ \\
    \bottomrule
  \end{tabular}
  \caption{Comparison operators}
  \label{tab:cmpops}
\end{table}

%\section{\lstinline!if!, \lstinline!elif!, and \lstinline!else!}
\section{Conditional statements}

The primary use for boolean values is to determine which branch in your code to
follow. This is accomplished using \lstinline{if} and \lstinline{else}, as shown
in the program below. \lstinline!elif! will be introduced later in this lab. All
three -- \lstinline!if!, \lstinline!elif!, and \lstinline!else! -- are generally
called \emph{conditional statements}.

\begin{lstlisting}[style=python]
x = 1
y = float(input("Please input a number: "))

if x == y:
    print("x and y are equal.")
else:
    print("x and y are not equal.")
print("When do I print?")
\end{lstlisting}

Try running the above program, putting in different numbers for \lstinline{y}.
If the number input is 1, then the first \lstinline!print! statement will
output. If not, then the second one will. The third one will output regardless.

The way this works is very simple: either the first \lstinline!print! statement
runs or the second \lstinline!print! statement runs, but never both. Which one
runs is determined by Python: if the boolean statement (called \emph{condition})
following the \lstinline!if! evaluates to \lstinline!True!, then Python will
run the indented code following the \lstinline!if! and then skip until after
the indented code of the \lstinline!else!.

However, if the \emph{condition} evaluates to \lstinline!False!, then the
indented code following the \lstinline!else! is run and Python skips the
indented code between \lstinline!if! and \lstinline!else!.

It is important to note that the code that follows \lstinline!if! or
\lstinline!else! \textbf{must} be indented.

See what happens when you run this compared to the other piece of code:
\begin{lstlisting}
x = 1
y = float(input("Please input a number: "))

if x == y:
    print("x and y are equal.")
else:
    print("x and y are not equal.")
    print("When do I print?")
\end{lstlisting}

%when Python sees an \lstinline{if} statement,
%it checks if the boolean statement immediately following it is \lstinline{True}.
%If it is, then it will run the indented section of code starting on the next
%line.  Once that section is done, it will skip past the indented
%\lstinline{else} block of code (if the \lstinline{else} exists -- it is
%optional), and continue running the rest of the program. If the statement was
%\lstinline{False}, then it will skip the block of indented code, run the
%\lstinline{else} block if it exists, and then continue running the rest of the
%program.

There are many uses for conditional statements, such as to ensure that a given
variable is not negative:

\begin{lstlisting}[style=python]
x = float(input("Please input a number: "))

if x < 0:
    x = 0 # sets x equal to 0 if x was less than 0

print("x = " + str(x))
\end{lstlisting}

You can perform other operations as part of a boolean statement, such as this convenient way to check if a number is even:

\begin{lstlisting}[style=python]
x = 5

if x % 2 == 0:
    print("x is even.")
else:
    print("x is odd.")
\end{lstlisting}

Remember that \lstinline!%! 
is the modulus operator: it gives you the remainder of the division.

When using \lstinline{if} and \lstinline{else}, you will generally be dealing
with user input. This is done using the function \lstinline{input}, which you
can see used in the above examples. When you use \lstinline{input} it will
display whatever string you pass to it, then pause while it waits for the user
to type something and then hit enter. It will give whatever was entered as a
string back to the variable that it is assigned to. We will be learning more
about strings in future labs, but for now just know that they are basically
groups of letters, like what you pass to a \lstinline{print} statement, and are
declared by surrounding something in quotes.

The main thing to know about strings for now is that they cannot be used as
numbers. This is why we use the \lstinline{float} function to convert the value
the user gives us into a number.

\begin{lstlisting}[style=python]
>>> 5.5 == "5.5"
False
>>> 5.5 == float("5.5")
True
\end{lstlisting}

% PROBLEMS HERE HERE HERE
% ``float cannot run'' -- they dont know what that means.
It is important to understand the order that things happen in a statement like

\lstinline{x = float(input("Please input a number: "))}

Though both \lstinline{x = } and \lstinline{float} appear first in the line, the
first statement to execute is \lstinline{input}. This is because
\lstinline{input} is inside of \lstinline{float}'s parentheses and is therefore
being passed as a parameter to \lstinline{float}. Therefore, \lstinline{float}
cannot run until \lstinline{input} is finished and has returned a value to be
used by \lstinline{float}. Similarly, \lstinline{x = } will not happen until
\lstinline{float} has finished converting the value into a number.

If you are comparing strings, then you do not need to go through the extra step
of converting the user's input into a number:

\begin{lstlisting}[style=python]
password = "hunter2"

user_pass = input("Please input the password: ")

if password == user_pass:
    print("Password is correct. Welcome!")
else:
    print("Invalid password.")
\end{lstlisting}

In some cases, it could be that there are multiple passwords. Try running the
following code:
\begin{lstlisting}
password = "hunter2"
also_password = "hunter3"
another_password = "hunter4"
user_pass = input("Please input the password: ")

if password == user_pass:
    print("This is one correct password.")
elif user_pass == also_password:
    print("Another correct password.")
elif user_pass == another_password:
    print("You entered a correct password.")
else:
    print("Wrong password.")
\end{lstlisting}

This introduced you to the \lstinline!elif! statement: When the condition
following \lstinline!if! turns out to be false, Python will then check the first
\lstinline!elif! statement. If that condition turns out to be true, it will run
the code following that \lstinline!elif! statement or move on to the next
\lstinline!elif!. Only if none of the conditions turned out to be true, the code
following \lstinline!else! will be run.

You can also nest the statements you just learned about. Try running the
following code, trying multiple values:
\begin{lstlisting}
x = float(input("Enter a value for x: "))
y = float(input("Enter a value for y: "))

if x > 0:
    if y > 0:
        print("Both are greater than 0.")
    else:
        print("x is greater than 0, but y is smaller or equal to 0")
else:
    print("x is smaller or equal to 0.")
\end{lstlisting}

\subsection{Summary}

\begin{itemize}
  \item Conditional statements look like this:
    \begin{lstlisting}
if condition:
    # some code to run
elif othercondition:
    # some other code to run
else:
    # alternative code if no condition was met
    \end{lstlisting}

  \item The \lstinline!elif! and \lstinline!else! sections are both optional
  \item \lstinline!elif! statements can be repeated as many times as you want.
  \item The conditions must be boolean statements.
  \item You can nest conditional statements.
  \item The code inside \lstinline!if!, \lstinline!elif!, and \lstinline!else!
    statements must be indented. Python will either show an error or behave very
    weirdly if you do not indent the code.
\end{itemize}

\subsection{Exercises}
\label{subsec:ifex}

\begin{description}
  \item[conversions.py] Use your \texttt{conversions.py} from last time and add a
    prompt asking the user whether to convert to Kelvin or Fahrenheit. It should
    look like this when it is run:

    \begin{lstlisting}[style=bash]
Please input the temperature in Celsius: 10
Please choose Kelvin (K) or Fahrenheit (F): F
You chose Fahrenheit.
Fahrenheit temperature: 50.0
    \end{lstlisting}
    \begin{lstlisting}[style=bash]
Please input the temperature in Celsius: 10
Please choose Kelvin (K) or Fahrenheit (F): K
You chose Kelvin.
Kelvin temperature: 283.15
    \end{lstlisting}
    \begin{lstlisting}[style=bash]
Please input the temperature in Celsius: 10
Please choose Kelvin (K) or Fahrenheit (F): E
You entered a letter I do not recognize.
    \end{lstlisting}

  \item[calculator.py] Write a small calculator that can compute arcsin, arccos,
    arctan and square root of a number. Use \lstinline!math.sqrt()!,
    \lstinline!math.asin()!, \lstinline!math.acos()!, and
    \lstinline!math.atan()!.

    \emph{Make sure to check for each function that the input is valid.} For
    example, for square root the input cannot be negative. For arcsin, the input
    must be between -1 and 1 inclusive. Try to figure out what the input must be
    for arccos and arctan yourself!

    \begin{lstlisting}[style=bash]
Enter a number to use: 16
Which operation? sqrt (s), arcsin (a), arccos (c), arctan (t): s
The square root of 16 is 4.0.
    \end{lstlisting}

\end{description}

\section{\lstinline{while} loops}
%Apply the above to while loops.
The syntax of a \lstinline{while} loop is very similar to that of an
\lstinline{if} statement, but instead of only running the indented block of code
once, the \lstinline{while} loop will continue running it until the given
boolean statement is no longer true.

\begin{lstlisting}[style=python]
x = 10

while x > 0:
    print(x)
    x = x - 1
\end{lstlisting}

The above program will print out the numbers 10 to 1. Try stepping through this
program on paper, writing out the value of \lstinline{x} at each time through
the loop. Then repeat for this modified version of the program:

\begin{lstlisting}[style=python]
x = 10

while x > 0:
    x = x - 1
    print(x)
\end{lstlisting}

This version of the program will print out the numbers 9 to 0. This might seem a
bit strange, since the condition of the loop says it will stop when
\lstinline{x} is no longer larger than 0. And yet, it prints out the value 0
before the loop ends. This is because the loop condition is only checked
whenever the end of the indented section is reached. If the condition is
\lstinline{True}, then the indented section will be executed again. If the
condition is \lstinline{False}, then the loop will end.

If the condition starts out \lstinline{False}, then the loop will never execute.
The following program will not print anything:

\begin{lstlisting}[style=python]
x = 0

while x > 0:
    x = x - 1
    print(x)
\end{lstlisting}

\lstinline{if} and \lstinline{else} can be combined with \lstinline{while}, as
shown below:

\begin{lstlisting}[style=python]
x = 10

while x > 0:
    if x % 2 == 0:
        print(str(x) + " is even.")
    else:
        print(str(x) + " is odd.")
    x = x - 1
\end{lstlisting}

Of course, they can be nested the other way around, too, with a
\lstinline!while! inside conditional statements.

There can also be infinite while loops. Try the following:
\begin{lstlisting}
while True:
    print("Printing forever")
\end{lstlisting}

Press \emph{Ctrl+C} to stop the execution of this.

\subsection{Summary}

\begin{itemize}
  \item Syntax:
    \begin{lstlisting}
while condition:
    # code to be repeated
    \end{lstlisting}

    This will repeat the indented code following the \lstinline!while! until the
    condition is not true anymore. It checks the condition first, then runs the
    indented code, then checks the condition again, etc. Thus, if the condition
    is wrong in the first place, it will never run.

  \item There can be infinite while loops.
\end{itemize}

\subsection{Exercises}
\label{subsec:whileex}

\begin{description}
  \item[fizzbuzz.py] Have the user enter a positive integer number. Then, print
    the numbers from 1 to that number each on a line. When the printed number is
    divisible by 3, print ``Fizz'', and when the number is divisible by 5, print
    ``Buzz'', and when it is divisible by both, print ``FizzBuzz''.

    Should look like this when run:
    \begin{lstlisting}[style=bash]
Enter a number: 16
1
2
3 Fizz
4
5 Buzz
6 Fizz
7
8
9 Fizz
10 Buzz
11
12 Fizz
13
14
15 FizzBuzz
16
    \end{lstlisting}
    \begin{lstlisting}[style=bash]
Enter a number: -1
Not a positive number!
    \end{lstlisting}

\end{description}

\section{Turtle}
%Cover how to use turtle, give an example of a program to use a while loop to draw a hexagon or something.
Some of the exercises for this lab will use Turtle, a simple graphics library. It can be accessed by using \lstinline{import turtle} in Python. From there you have access to a group of functions for controlling the ``turtle'', a simple arrow that moves around at your command, drawing a line where it goes. The primary commands to control the turtle are shown in Table \ref{tab:turtle}.

\begin{table}[h]
  \centering
  \begin{tabular}{ll}
    \toprule
    Operator & What it does\\
    \midrule
    \lstinline!turtle.forward(x)! & move the turtle forward \lstinline!x! pixels \\
    \lstinline!turtle.left(x)! & turn the turtle left \lstinline!x! degrees \\
    \lstinline!turtle.right(x)! & turn the turtle right \lstinline!x! degrees \\
    \bottomrule
  \end{tabular}
  \caption{Turtle commands}
  \label{tab:turtle}
\end{table}

Combining these commands will let you draw potentially complex shapes. For example, the following program will draw a Hexagon.

\begin{lstlisting}[style=python]
import turtle

turtle.forward(100)
turtle.left(60)
turtle.forward(100)
turtle.left(60)
turtle.forward(100)
turtle.left(60)
turtle.forward(100)
turtle.left(60)
turtle.forward(100)
turtle.left(60)
turtle.forward(100)
turtle.left(60)
\end{lstlisting}

However, this code is a big longer than it needs to be. Let's clean it up a bit using \lstinline{while}.

\begin{lstlisting}[style=python]
import turtle

sides = 6
angle = 360 / 6
counter = 0

while counter < 6:
    turtle.forward(100)
    turtle.left(60)
    counter = counter + 1
\end{lstlisting}


\section{\texttt{.format()}}

\pagebreak

\section{Submitting}

Files to submit:
\begin{itemize}
  \item conversions.py (see Section~\ref{subsec:ifex})
  \item calculator.py (see Section~\ref{subsec:ifex})
  \item fizzbuzz.py (see Section~\ref{subsec:whileex})
\end{itemize}

You may submit your code as either a tarball (instructions below) or as a .zip
file. Either one should contain all files used in the exercises for this lab.
The submitted file should be named either
\texttt{cse107\_firstname\_lastname\_lab2.zip} or
\texttt{cse107\_firstname\_lastname\_lab2.tar.gz} depending on which method you
used.

For Windows, use a tool you like to create a \texttt{.zip} file. The TCC computers should
have \texttt{7z} installed.

\begin{center}
  \textbf{Upload your tarball or .zip file to Canvas.}
\end{center}

\subsection{Linux}

Tar is used much the same way that Zip is used in Windows: it combines many files and/or directories into a single file. Gzip is used in Linux to compress a single file, so the combination of Tar and Gzip do what Zip does. However, Tar deals with Gzip for you, so you will only need to learn and understand one command for zipping and extracting.

In the terminal (ensure you are in your \texttt{lab1} directory), type the following command, replacing \texttt{firstname} and \texttt{lastname} with your first and last names:

\begin{lstlisting}[style=bash]
tar czvf cse107_firstname_lastname_lab2.tar.gz *.py
\end{lstlisting}

This creates the file \texttt{cse107\_firstname\_lastname\_lab1.tar.gz} in the directory. The resulting archive, which includes every python file in your \texttt{lab1} directory, is called a tarball. 

To check the contents of your tarball, run the following command:

\begin{lstlisting}[style=bash]
tar tf cse107_firstname_lastname_lab2.tar.gz *.py
\end{lstlisting}

You should see a list of your Python source code files.

\end{document}
