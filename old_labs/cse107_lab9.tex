% LAB 9: Dictionaries
% CSE/IT 107: Introduction to Programming
% New Mexico Tech
% 
% Prepared by Scott Chadde 
% Fall 2013

\documentclass[12pt]{article}



% Allows for bold tele-type
\DeclareFontShape{OT1}{cmtt}{bx}{n}{
  <5><6><7><8><9><10><10.95><12><14.4><17.28><20.74><24.88>cmttb10}{}


% AMS math formatting
\usepackage{amsmath}
\usepackage{amssymb}
\usepackage{amsthm}     
\usepackage{textcomp}
% Indention formatting
\setlength{\parindent}{0.0in}
\setlength{\parskip}{0.05in}

% Margin support
\usepackage[margin=1in]{geometry}

% Header/footer support
\usepackage{fancyhdr}
\pagestyle{fancy}
\fancyhead{}
\fancyhead[L]{CSE 107}
%%%%%%%%%%%%%%%%%%%%%%%%%%%%%%%%%%%%%%%%%%%%%%%%%%%%%%%%%%%%%%%%%%%%%%%%%%%
% Replace with title for lab
\fancyhead[R]{Lab 9: Dictionaries}
%%%%%%%%%%%%%%%%%%%%%%%%%%%%%%%%%%%%%%%%%%%%%%%%%%%%%%%%%%%%%%%%%%%%%%%%%%%

% Color support
\usepackage{color}
\definecolor{light-gray}{gray}{0.9}
% Outline support
%\usepackage{outlines}
% URL formatting
\usepackage{}\usepackage{url}
% Image/graphic support
\usepackage{graphicx}
% Supports in-document hyperlinks
\usepackage[pdfborder=0in]{hyperref}
% List compression
\usepackage{mdwlist}
\usepackage{enumerate}
\usepackage{enumitem}
% Support long tables across pages
\usepackage{longtable}
\setlength{\LTcapwidth}{6in}
\usepackage{array}
\usepackage{multirow}
% Supports code formatting/highlighting
\usepackage{listings}
\lstset{basicstyle=\ttfamily, keywordstyle=\bfseries\color{black}, stringstyle=\color{black}, commentstyle=\color{blue}, showstringspaces=false, numbers=none}
\lstdefinestyle{bash}{language=bash, backgroundcolor=\color{light-gray}}
\lstdefinestyle{c}{language=C, frame=single}

%%%%%%%%%%%%%%%%%%%%%%%%%%%%%%%%%%%%%%%%%%%%%%%%%%%%%%%%%%%%%%%%%%%%%%%%%%%
% Replace with title for lab
\title{Lab 9: Dictionaries}
%%%%%%%%%%%%%%%%%%%%%%%%%%%%%%%%%%%%%%%%%%%%%%%%%%%%%%%%%%%%%%%%%%%%%%%%%%%
\author{CSE/IT 107}
\date{}

\begin{document}

\maketitle


\hrule
%%%%%%%%%%%%%%%%%%%%%%%%%%%%%%%%%%%%%%%%%%%%%%%%%%%%%%%%%%%%%%%%%%%%%%%%%%%
%%%%%%%%%%%%%%%%%%%%%%%%%%%%%%%%%%%%%%%%%%%%%%%%%%%%%%%%%%%%%%%%%%%%%%%%%%%
\section*{Introduction}

The purpose of this lab is to give you some experience with dictionaries, a very nice data structure.

\section*{Reading}

Chapter 11 in Think Python: How to Think Like a Computer Scientist (Think)

\section*{Coding Conventions}

Follow PEP-8 recommendations for your code. 

\section*{Problems}

Make sure your source code files are appropriately named. Make sure your code has a main function; use \texttt{boilerplate.py} you created in Lab 1.

\emph{Think Python} is available from \url{http://www.greenteapress.com/thinkpython/}. Page numbers and exercise numbers refer to those found in the PDF file available at the website.


\begin{enumerate}
       \item Exercise 11.1 \emph{Think Python}, page 102. Name your source code \texttt{think\_11-1.py}. The file \texttt{words.txt} is available at \url{http://www.greenteapress.com/thinkpython/code/words.txt}.
        

       \item Exercise 11.2 \emph{Think Python}, page 103. Name your source code \texttt{think\_11-2.py}.

       \item Exercise 11.3 \emph{Think Python}, page 104. Name your source code \texttt{think\_11-3.py}.


       \item Exercise 11.4 \emph{Think Python}, page 105. Name your source code \texttt{think\_11-4.py}.


       \item Exercise 11.6 \emph{Think Python}, page 106. Name your source code \texttt{think\_11-6.py}. Run the memo version of the Fibonacci sequence and print out the first 500 values of the Fibonacci sequence to a file name \texttt{fib500.txt}, one value per line.  

        \item Exercise 11.9 \emph{Think Python}, page 111.

        \item Rewrite the program Rock-paper-scissors-lizard-Spock from lab 4 using dictionaries. Name your source code file \texttt{rock-spock-dict.py}.

        \item This problem is from Google's Python Class (\url{https://developers.google.com/edu/python/}):

"Read in the file specified on the command line.
Do a simple split() on whitespace to obtain all the words in the file.
Rather than read the file line by line, it's easier to read
it into one giant string and split it once.

Build a "mimic" dict that maps each word that appears in the file
to a list of all the words that immediately follow that word in the file.
The list of words can be be in any order and should include
duplicates. So for example the key "and" might have the list
["then", "best", "then", "after", ...] listing
all the words which came after "and" in the text.
We'll say that the empty string is what comes before
the first word in the file.

With the mimic dict, it's fairly easy to emit random
text that mimics the original. Print a word, then look
up what words might come next and pick one at random as
the next work.

Use the empty string as the first word to prime things.
If we ever get stuck with a word that is not in the dict,
go back to the empty string to keep things moving.

Note: the standard python module 'random' includes a
random.choice(list) method which picks a random element
from a non-empty list."

\begin{verbatim}
import random
import sys


def mimic_dict(filename):
  """Returns mimic dict mapping each word to list of words which follow it."""
  # +++your code here+++
  return


def print_mimic(mimic_dict, word):
  """Given mimic dict and start word, prints 200 random words."""
  # +++your code here+++
  return



# Provided main(), calls mimic_dict() and mimic()
def main():
  if len(sys.argv) != 2:
    print 'usage: ./mimic.py file-to-read'
    sys.exit(1)

  dict = mimic_dict(sys.argv[1])
  print_mimic(dict, '')


if __name__ == '__main__':
  main() 
\end{verbatim}

Use the file \texttt{mimic.py} and read in the files \texttt{alice.txt} (Alice in Wonderland) and \texttt{mphg.txt} (Monty Python and the Holy Grail), which are all available on Moodle. 

You are building a dictionary that looks like this:

\begin{verbatim}
{'a': ['hat', 'tree', 'swallow'], 'hat': ['flew', 'went']}
\end{verbatim}

To figure out how to create the dictionary try the
following in an interpreter:

\begin{verbatim}
>>> d = {}
>>> d['a'] = [1]
>>> print d
>>> d.get('a').append(2)
>>> print d
\end{verbatim}

Once
you get the program working for 200 words, modify so it will create a
mimic of the whole text.

For more information about this problem, read \url{http://en.wikipedia.org/wiki/Markov_chain} and \url{http://joshmillard.com/garkov/} 

\end{enumerate}

% future labs
% to_lower and to_upper & ascii table -- chr and ord

%\item Shuffle DNA - use Knuth's algorithm

% codon frequencys - build a list, sort it, and get the count of codons

% write reverse with slice and dice methodology

% extract the first base-pair of the codon, second -base pair, and third base pair (tuples). Get the GC content.

\section*{Submission}

Create a tarball of your *.py files and fib1000.txt

\begin{lstlisting}[style=bash]
tar czvf cse107_firstname_lastname_lab9.tar.gz *.py fib500.txt
\end{lstlisting}


To check the contents of your tarball, run the following command:

\begin{lstlisting}[style=bash]
tar tf cse107_firstname_lastname_lab9.tar.gz
\end{lstlisting}

You should see a list of your Python source code files.

Upload your tarball in Moodle before the start of you next lab.

\end{document}
