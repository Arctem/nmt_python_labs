% LAB 3: Functions
%
% CSE/IT 107: Introduction to Programming
% New Mexico Tech
%
% Prepared by Russell White and Christopher Koch
% Spring 2015

% - Functions
%   - Positional and Optional Arguments
%   - Recursion
% - Modules
%   - import statements
%   - main() boilerplate
% - Comments and PEP-8 style
% - Keywords: def
\documentclass[11pt]{cselabheader}
\fancyhead[R]{Lab 3: Functions}
\title{Lab 3: Functions}

\begin{document}
\pagenumbering{roman}
\maketitle

\begin{figure}[H]
  \centering
  \includegraphics[width=0.85\textwidth]{img/xkcd_python.png}
  \caption{xkcd 353: Python (Source: \url{http://xkcd.com/353})}
\end{figure}

\pagebreak
\hrule
\begin{quotation}
``If you don't think carefully, you might believe that programming is just
typing statements in a programming language.''
\end{quotation}
\begin{flushright}
  --- W. Cunningham
\end{flushright}

\begin{quotation}
``Only ugly languages become popular. Python is the exception.''
\end{quotation}
\begin{flushright}
  --- Donald Knuth
\end{flushright}

\begin{quotation}
``The time you enjoy wasting is not wasted time.''
\end{quotation}
\begin{flushright}
  --- Bertrand Russell
\end{flushright}
\hrule

\tableofcontents

\pagebreak
\pagenumbering{arabic}

\section{Introduction}
\label{sec:intro}

In the previous lab, we showed you simple control flow and how to repeat a piece
of code using \pythoninline{while}. In this lab, we will be learning how to
break a lot of code into smaller, reusable pieces called \emph{functions}.

\section{Making Calculations Shorter}
\label{sec:calc}

We showed you simple Python operators such as \pythoninline{+},
\pythoninline{-}, \pythoninline{*}, \pythoninline{%},
etc in lab 1. There is a small extension to these that you can use to update a
variable:

\begin{pyconcode}
>>> x = 5
>>> x += 3 # same as x = x + 3
>>> x
8
\end{pyconcode}

The available assignment operators are:
\begin{multicols}{2}
\begin{itemize}
  \item \pythoninline{+=} -- addition
  \item \pythoninline{-=} -- subtraction
  \item \pythoninline{*=} -- multiplication
  \item \pythoninline{/=} -- division
  \item \pythoninline{//=} -- integer division
  \item \pythoninline{%=} -- remainder
  \item \pythoninline{**=} -- exponentiation
\end{itemize}
\end{multicols}
They each correspond to the non-assignment version.

\pagebreak
\section{\protect\pythoninline{def}: Functions}
\label{sec:funcs}

So far, we have used functions such as \pythoninline{print()} and
\pythoninline{math.sqrt()}, but we have not yet written our own function. 

But first, let's talk about why to write functions. Some reasons:
\begin{itemize}
  \item Instead of writing the same code again, we can just call a function
    containing the code again. (Functions are \emph{reusable}.)
  \item Functions allow us to break our programs into many smaller pieces. This
    also allows us to easily think about each small piece in detail.
  \item Functions allow us to test small parts of our programs while not
    affecting other parts of the program -- this reduces errors in our code.
\end{itemize}

A Python function is simply a ``container'' for a sequence of Python statements
that do some task. Usually, a function does one task and one task only, but it
does it really well. Here's the general form of how to write a function:

\begin{python3code}
def function_name(arg0, arg1, ...):
    # block of code
\end{python3code}

A function can have \emph{zero or more} arguments. For example:

\begin{pyconcode}
>>> def pirate_noises():
...     i = 1
...     while i <= 4:
...         print("Arr!")
...         i += 1
...
\end{pyconcode}

To call this function:

\begin{pyconcode}
>>> pirate_noises()
Arr!
Arr!
Arr!
Arr!
\end{pyconcode}

To call a function, use its name followed by parentheses which contain
comma-separated parameters:

\begin{python3code}
function_name(param0, param1, ...)
\end{python3code}

\begin{itemize}
  \item You must use parentheses both in the function definition and the function
    call, even if there are zero arguments.
  \item The parameter values are substituted for the corresponding arguments to
    the function. I.e. the value of parameter param0 is substituted for argument
    arg0, param1 is substituted for arg1, and so forth.
\end{itemize}

For example:

\begin{pyconcode}
>>> def grocer(num_fruits, fruit_kind):
...     print('Stock: {} cases of {}'.format(num_fruits, fruit_kind))
...
>>> grocer(37, 'kale')
Stock: 37 cases of kale
>>> grocer(0, 'bananas')
Stock: 0 cases of bananas
\end{pyconcode}

\subsection{\protect\pythoninline{return}: Giving back values from a function}

When we used functions from the math module, we were always able to assign the
result of a function to a variable or to print it. For example:

\begin{pyconcode}
>>> import math
>>> x = math.sqrt(16)
>>> print(x)
4.0
\end{pyconcode}

So how do we get a function to give back a value (\emph{return} a value)? We use
the return statement:

\begin{pyconcode}
>>> def square(x):
...     return x**2
...
>>> y = square(5)
>>> print(y)
25
>>> square(4.3)
18.49
\end{pyconcode}

As soon as a \pythoninline{return} statement is reached, the function stops
executing and just returns the value given to it. Any subsequent statements that
are part of the function will be omitted. For example:

\begin{pyconcode}
>>> def wage(hours, base_rate):
...     if hours > 40:
...         ot_pay = (hours - 40) * base_rate * 1.5
...         return base_rate * 40 + ot_pay
...     pay = hours * base_rate
...     return pay
...
>>> wage(40, 10)
400
>>> wage(50, 10)
550
\end{pyconcode}

\begin{itemize}
  \item You can omit the expression after the return and just use a statement of
    this form:

    \begin{python3code}
return
    \end{python3code}

    In this case, the special value \pythoninline{None} is returned from the
    function.

  \item If Python executes your function body and never encounters a
    \pythoninline{return} statement, the effect is the same as a
    \pythoninline{return} with no value at the end of the function body: the
    special value \pythoninline{None} is returned.
\end{itemize}

\subsection{Summary}
\label{subsec:funcs.sum}

\begin{itemize}
  \item Function definition syntax:

    \begin{python3code}
def function_name(arg0, arg1, ...):
    # block of code
    \end{python3code}

    Function call syntax:

    \begin{python3code}
function_name(param0, param1, ...)
    \end{python3code}

  \item A function may take zero or more arguments.

  \item A function returns one value. (If the programmer does not specify a
    value, the special value \pythoninline{None} is returned.)

  \item A good resource:

\begin{center}
  \url{https://docs.python.org/3.4/tutorial/controlflow.html#defining-functions}
\end{center}

\end{itemize}

\subsection{Exercises}
\label{subsec:funcs.ex}

\begin{ex}[mathutils.py] Write a Python file that contains the following functions:

    \begin{itemize}
      \item \pythoninline{sum_numbers}: given an integer $n$, returns the sum of
        the integers between one and $n$.
      \item \pythoninline{right_triangle_hypotenuse}: given the lengths of each
        of the two legs of a right triangle, returns the length of the
        hypotenuse.  (Hint: remember the Pythagorean theorem.)
      \item \pythoninline{triangle_area}: given the base width and height of a
        triangle, returns the area of the triangle.
      \item \pythoninline{triangle_angle}: given two angles' value in degrees,
        finds the value of the third angle of a triangle.
    \end{itemize}
\end{ex}

\begin{ex}[calls.py] You buy an international calling card to Germany. The calling
    card company has some special offers.

    \begin{enumerate}[(a)]
      \item If you charge your card with less than \$10, you don't get anything
        extra.
      \item For a less than \$25 charge, you get \$3 of extra phone time.
      \item For a less than \$50 charge, you get \$8 of extra phone time.
      \item For a less than \$100 charge, you get \$20 of extra phone time.
      \item For a more than \$100 charge, you get \$25 of extra phone time.
    \end{enumerate}

    Write a function that takes the value the user wants to 
\end{ex}

\pagebreak
\section{Conventions}
\label{sec:pep8}

In order to make code more readable, we will start requiring you to comment your
code and follow a style guide. Style guides are often used to make code easy to
read, especially if multiple people are working on a project together. If left
to their own devices, most people start conforming to their own style guide
anyway just by preferring a certain way to write something over another. For
example, common parameter of style guides is the use of a certain number of
spaces for indentation. Also, some people put spaces before each colon, and some
people do not.

\subsection{Style Guide}

We will be using PEP 8 (Python Enhancement Proposal 8 -- Style Guide for Python
Code) found at
\begin{center}
  \url{https://www.python.org/dev/peps/pep-0008/}
\end{center}

Some of the highlights:
\begin{itemize}
  \item 4 space indentation
  \item Function names should be all-lowercase with words separated by underscores.
  \item File/module/package names should have short, all-lowercase names.
  \item Comment your code with useful information. For example,

    \begin{python3code}
x = x + 1 # Increment x
    \end{python3code}

    should be avoided. It is obvious that \pythoninline{x} is being incremented.
    Instead, if you think a comment will improve code comprehension, the
    following can be useful:

    \begin{python3code}
x = x + 1 # Compensate for border
    \end{python3code}

  \item Avoid whitespace where it does not help code legibility. Never put a
    space between a function name and the parentheses when calling a function.

    \begin{python3code}
if x == 4: # do this
    print(x, y)

if x == 4 : # don't do this
    print ( x , y )
    \end{python3code}
\end{itemize}

\pagebreak
\subsection{Commenting Functions}

For commenting on functions, we will be using PEP 257 (Docstring Conventions)
found at
\begin{center}
  \url{https://www.python.org/dev/peps/pep-0257/}
\end{center}

\begin{center}
\bfseries \large You will be required to put a \emph{docstring} at the beginning
of every function that you code from now on.
\end{center}
A docstring is a comment immediately
following the function definition enclosed by triple-double-quotes (\texttt{"""}).

The highlights:
\begin{itemize}
  \item For short functions, do this:

    \begin{python3code}
def midpoint(a, b):
    """Find and return the midpoint of the given a and b."""
    return (a+b)/2
    \end{python3code}

  \item For larger functions or for a longer explanation, follow this style:

    \begin{python3code}
def calculate_weekly_pay(pay_rate, hours, tax_rate):
    """Calculate the net pay after taxes given the number of hours worked 
    in a week, a pay rate, and a flat tax rate.

    Arguments:
    pay_rate -- rate of pay
    hours -- number of hours worked in one week
    tax_rate -- flat tax rate (for example, 0.15 for 15%)
    """
    pay_before_taxes = hours * pay_rate

    # Add overtime payment if necessary
    if hours > 40:
        pay_before_taxes += (hours - 40) * pay_rate * 0.5

    pay_after_taxes = pay_before_taxes * (1 - tax_rate)
    return pay_after_taxes
    \end{python3code}

\end{itemize}

\pagebreak
\section{Modules}
\label{sec:modules}

A program that is saved in a text file and then run with Python is called a
\emph{script}. As your programs get longer, you may want to split them into
multiple files for easier legibility, maintenance, or abstraction. To do this,
Python supports a way to put definitions (of functions) in a file and use them
in another script or in the interpreter. A file like this is called a
\emph{module}. An example for such a module is the math module we have used
previously. The definitions from a module can be \emph{imported} into other
scripts.

A module is just a file with Python statements in it. A module takes the name of
the file that it is in. For example, imagine we have a file
\texttt{circlemath.py} in our current working directory:

\begin{python3code}
import math # importing the math module

def area(radius):
    """Find and return the area of a circle (float) given the radius"""
    return math.pi * radius**2

def circumference(radius):
    """Find and return the circumference of a circle (float) given the radius"""
    return 2 * math.pi * radius
\end{python3code}

Then, open the interpreter and import the module:

\begin{pyconcode}
>>> import circlemath
>>> circlemath.circumference(5) # have to use modulename.attributename
31.41592653589793
>>> circlemath.area(5)
78.53981633974483
>>> circlemath.__name__
'circlemath'
\end{pyconcode}

Every module has a \pythoninline{__name__} attribute that contains the name of
the module, except in one special case that we will see soon.

There are other kinds of import statements with different effects:

\begin{pyconcode}
>>> from circlemath import area
>>> area(5) # can just use the attribute imported without modulename.
78.53981633974483
>>> circumference(5) # not defined, because not imported
Traceback (most recent call last):
  File "<input>", line 1, in <module>
NameError: name 'circumference' is not defined
\end{pyconcode}

\begin{pyconcode}
>>> from circlemath import area, circumference # use commas to separate multiple
>>> area(5)
78.53981633974483
>>> circumference(5)
31.41592653589793
\end{pyconcode}

\begin{pyconcode}
>>> from circlemath import * # import all definitions made
>>> area(5)
78.53981633974483
>>> circumference(5)
31.41592653589793
\end{pyconcode}

\begin{pyconcode}
>>> from circlemath import area as carea
>>> carea(5)
78.53981633974483
\end{pyconcode}

The last option is rather frowned upon; we would prefer you to use
\pythoninline{circlemath.area} instead as it is more descriptive.

You may also rename a module upon importing it:

\begin{pyconcode}
>>> import circlemath as cmath
>>> cmath.area(5)
78.53981633974483
\end{pyconcode}


\begin{infobox}{Convention (PEP 8)}
  Always put \pythoninline{import} statements at the beginning of a file in the
  following order:
  \begin{enumerate}
    \item Built-in standard library modules
    \item Third-party modules (e.g. matplotlib)
    \item Self-written modules
  \end{enumerate}
  Put a blank line between each group of imports.
\end{infobox}

\subsection{Using Modules as Scripts and Boilerplate}

If there is code in your module that is not a function definition, Python will
run it just once when the module is imported. For example, take the file
\texttt{mid.py}:

\begin{python3code}
def midpoint(a, b):
    """Find and return the midpoint of two numbers."""
    return (a+b)/2

print('Midpoint of {} and {} is {:.2f}.'.format(5, 10, midpoint(5, 10)))
\end{python3code}

\begin{pyconcode}
>>> import mid
Midpoint of 5 and 10 is 7.50.
\end{pyconcode}

The same will happen if you run the module like a script:

\begin{bashcode}
$ python3 mid.py
Midpoint of 5 and 10 is 7.50.
\end{bashcode}

However, this can be rather annoying to deal with, for example if you wrote a
script with a bunch of functions a while ago and you just want to use the
functions you wrote, but do not want the other code to run when you're using
them -- you just want the functions. You can achieve this!

When you run a module as a script, the modules \pythoninline{__name__} attribute
is not set to the module's name, but to \pythoninline{"__main__"}. Hence, the
following code in \texttt{mid.py} will do the following:

\begin{python3code}
def midpoint(a, b):
    """Find and return the midpoint of two numbers."""
    return (a+b)/2

print(__name__) # just for demonstration, do not put this in real programs

if __name__ == "__main__":
    print('Midpoint of {} and {} is {:.2f}.'.format(5, 10, midpoint(5, 10)))
\end{python3code}

\begin{pyconcode}
>>> import mid
mid
\end{pyconcode}

\begin{bashcode}
$ python3 mid.py
__main__
Midpoint of 5 and 10 is 7.50.
\end{bashcode}

We call this specific if-statement referencing the \pythoninline{__name__}
variable boilerplate code.

\begin{infobox}{Boilerplate Requirement}
  Please put \textbf{any} code that is not part of a function inside the
  boilerplate if-statement so that \emph{every} one of your scripts can also be
  used as a module. This will be required for every lab from now on.

  Wrong:

  \begin{python3code}
print("Hello World!")
  \end{python3code}

  Right:

  \begin{python3code}
if __name__ == "__main__":
    print("Hello World!")
  \end{python3code}
\end{infobox}

\subsection{The \protect\pythoninline{dir()} Function}

Use the \pythoninline{dir()} function to get a list of every attribute --
variables, functions, and other things you do not know about yet -- that is part
of a module. For example for the circlemath module we wrote earlier:

\begin{pyconcode}
>>> import circlemath
>>> dir(circlemath)
['__name__', 'area', 'circumference']
>>> import mid
>>> dir(mid)
['__name__', 'midpoint']
>>> dir(math)
['__doc__', '__file__', '__loader__', '__name__', '__package__', '__spec__', 'acos',
'acosh', 'asin', 'asinh', 'atan', 'atan2', 'atanh', 'ceil', 'copysign', 'cos', 'cosh',
'degrees', 'e', 'erf', 'erfc', 'exp', 'expm1', 'fabs', 'factorial', 'floor', 'fmod',
'frexp', 'fsum', 'gamma', 'hypot', 'isfinite', 'isinf', 'isnan', 'ldexp', 'lgamma',
'log', 'log10', 'log1p', 'log2', 'modf', 'pi', 'pow', 'radians', 'sin', 'sinh', 'sqrt',
'tan', 'tanh', 'trunc']
\end{pyconcode}

\subsection{Summary}
\label{subsec:modules.sum}

\begin{itemize}
  \item Syntax:

    \begin{python3code}
import modname1, modname2 # use: modname1.attributename
import modname3 as othermodname # use: othermodname.attributename
from modname import attribute1, attribute2 # use: attribute1, attribute2
from modname import * # use: attributename
from modname import attribute as otherattribute # use: otherattribute
    \end{python3code}

    Attributes can be functions defined in the module or variables defined in
    the module. A special attribute is \pythoninline{__name__}, which contains
    the name of the module unless the module is executed as a script.

  \item Every script is a module whose name is the filename of the script.

  \item If a module is imported, its \pythoninline{__name__} attribute is a
    variable that contains the module name. If a module is run as a script, its
    \pythoninline{__name__} attribute contains \pythoninline{"__main__"}. With
    this, you can have code that only runs if a module is run like a script.

    We call that boilerplate code:

    \begin{python3code}
if __name__ == "__main__":
    # stuff that only runs if module is run as a script
    \end{python3code}

  \item Use \pythoninline{dir()} to get a list of every attribute -- every
    variable and function (and other things that you do not know about yet) --
    that is part of a particular module.

  \item A good resource:

    \begin{center}
\url{https://docs.python.org/3.4/tutorial/modules.html}
    \end{center}
\end{itemize}

\subsection{Exercises}
\label{subsec:modules.ex}

\pagebreak
\section{Advanced Functions}
\label{sec:adv}

\subsection{Positional and Optional Arguments}
\label{subsec:adv.args}

\subsection{Recursion}
\label{subsec:adv.recursion}

\subsection{Summary}
\label{subsec:adv.sum}

\subsection{Exercises}
\label{subsec:adv.ex}

\begin{ex}[sum.py] Write a recursive function that computes the sum of all the
  numbers from 1 to $n$, where $n$ is the given parameter.
\end{ex}

\begin{ex}[expo.py] Write a recursive function that takes in a base and an
  exponent and returns the value of that base taken to that exponent. You are
  not allowed to use \lstinline{**} or any built-in exponentiation functions.
\end{ex}

\pagebreak
\section{Submitting}

You should submit your code as a tarball. It should contain all files
used in the exercises for this lab. The submitted file should be named
\begin{center}
  \texttt{cse107\_firstname\_lastname\_lab3.tar.gz}
\end{center}

\begin{center}
  \textbf{Upload your tarball to Canvas before the deadline.}
\end{center}

\listoftheorems

\end{document}
