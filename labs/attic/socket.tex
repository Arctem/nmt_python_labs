% LAB 9: Sockets
% 
% CSE/IT 107: Introduction to Programming
% New Mexico Tech
% 
% Prepared by Russell White and Christopher Koch
% Fall 2014
\documentclass[11pt]{cselabheader}

%%%%%%%%%%%%%%%%%% SET TITLES %%%%%%%%%%%%%%%%%%%%%%%%%
\fancyhead[R]{Lab 9: Introduction to Objects}
\title{Lab 9: Introduction to Objects}

\begin{document}

\pagenumbering{roman}
\maketitle

\hrule
\begin{quotation}
``Today, most software exists not to solve a problem, but to interface with other
software.''
\end{quotation}
\begin{flushright}
--- Ian O. Angell
\end{flushright}

\begin{quotation}
``Computer Science is no more about computers than astronomy is about
telescopes.''
\end{quotation}
\begin{flushright}
--- Edsger W. Dijkstra
\end{flushright}

\begin{quotation}
``If people never did silly things, nothing intelligent would ever get done.''
\end{quotation}
\begin{flushright}
--- Ludwig Wittgenstein
\end{flushright}

\hrule
\section*{Introduction}
\addcontentsline{toc}{section}{Introduction}

\pagebreak
\tableofcontents

\pagebreak
\pagenumbering{arabic}
\section{Classes}

\section{Exercises}

\subsection{Review}

\begin{ex}[piglatin.py] Write a program that translates a file from English to
  pig latin.

  The rules for pig latin are as follows:

  For a word that begins with consonant sounds, the initial consonant or
  consonant cluster is moved to the end of the word and ``ay'' is added as a
  suffix:
  \begin{itemize}
    \item ``happy'' $\to$ ``appyhay''
    \item ``glove'' $\to$ ``oveglay''
  \end{itemize}

  For words that begin with vowels, you add ``way'' to the end of the word:
  \begin{itemize}
    \item ``egg'' $\to$ ``eggway''
    \item ``inbox'' $\to$ ``inboxway''
  \end{itemize}

  For your program, you \emph{must} write a function that takes in one
  individual word and returns the translation to pig latin. Write another
  function that takes a string, which may be sentences (may contain the
  characters ``a-zA-Z,.-;!?()'' and space), and returns the translation of the
  sentence to pig latin. Strip out any punctuation. For example, ``Hello, how
  are you?'' would translate into ``elloHay owhay areway ouyay''.

  The user must be able to specify the filename for the file to be translated
  and the filename that the program should write to. For example:

  \begin{bashcode}
$ cat test.txt
Hello, how are you?
$ python3 piglatin.py
Enter English filename >>> test.txt
Enter filename to write to >>> test_piglatin.txt
Done.
$ cat test_piglatin.txt
elloHay owhay areway ouyay
  \end{bashcode}
\end{ex}

\begin{ex}[luhns.py] Luhn's algorithm
    (\url{http://en.wikipedia.org/wiki/Luhn_algorithm}) provides a quick way to
    check if a credit card is valid or not. The algorithm consists of three
    steps:

    \begin{enumerate}
      \item Starting with the second to last digit (ten's column digit),
        multiply every other digit by two.
      \item Sum all the digits of the resulting number.
      \item If the total sum modulo by 10 is zero, then the card is valid;
        otherwise it is invalid.
    \end{enumerate}

    For example, to check that the Diners Club card 38520000023237 is valid, you
    would start at 3, double it and double every other digit to give, writing
    the credit card number as separate digits: 
    \begin{IEEEeqnarray*}{RCCRCCRCCRCCRCCRCCRCCRCCRCCRCCRCCRCCR}
3 &~& 8 &~& 5  &~& 2 &~& 0 &~& 0 &~& 0 &~& 0 &~& 0 &~& 2 &~& 3 &~& 2 &~& 3 &~& 7\\
6 &~& 8 &~& 10 &~& 2 &~& 0 &~& 0 &~& 0 &~& 0 &~& 0 &~& 2 &~& 6 &~& 2 &~& 6 &~& 7
    \end{IEEEeqnarray*}
    Next you would sum all the digits of the resulting number:
    \[ 6 + 8 + (1 + 0) + 2 + 0 + 0 + 0 + 0 + 0 + 2 + 6 + 2 + 6 + 7 = 40 \]

    Note that for 10, you also sum its digits: $(1 + 0)$.

    The last step is to check if $40$ modulus 10 is equal to 0, which is true.
    So the card is valid.

    Write a program that implements Luhn's Algorithm for validating credit
    cards. It should ask the user to enter a credit card number and tell the
    user whether it is valid or not. 

    \emph{There must be a separate function that takes in a card number and
    validates it.}
  \end{ex}
\section{Submitting}

You should submit your code as a tarball. It should contain all files
used in the exercises for this lab. The submitted file should be named
\begin{center}
  \texttt{cse107\_firstname\_lastname\_lab9.tar.gz}
\end{center}

\begin{center}
  \textbf{Upload your tarball to Canvas.}
\end{center}

\listoftheorems


\end{document}
