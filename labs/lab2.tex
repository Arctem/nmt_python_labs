% LAB 2: Basic Flow Control
% 
% CSE/IT 107: Introduction to Programming
% New Mexico Tech
% 
% Prepared by Russell White and Christopher Koch
% Fall 2014
\documentclass[11pt]{cselabheader}
%%%%%%%%%%%%%%%%%% SET TITLES %%%%%%%%%%%%%%%%%%%%%%%%%
\fancyhead[R]{Lab 2: Basic Flow Control}
\title{Lab 2: Basic Flow Control}

\begin{document}

\pagenumbering{roman}
\maketitle

\hrule

\begin{quotation}
``When you come to a fork in the road, take it."
\end{quotation}
\begin{flushright}
--- Attributed to Yogi Berra
\end{flushright}

\begin{quotation}
``Simplicity is the ultimate sophistication.''
\end{quotation}
\begin{flushright}
--- Leonardo Da Vinci
\end{flushright}

\begin{quotation}
``How do we convince people that in programming simplicity and clarity -- in
short: what mathematicians call elegance -- are not a dispensable luxury, but
a crucial matter that decides between success and failure?''
\end{quotation}
\begin{flushright}
--- Edsger Dijkstra
\end{flushright}

\hrule

\section*{Introduction}
\addcontentsline{toc}{section}{Introduction}

The purpose of this lab is to introduce you to the fundamentals of what
programmers call flow control. In the previous lab, we showed you how to do
basic calculations in Python. For example, we had you convert temperature from
Celsius to Fahrenheit and Kelvin.

What if the user of your conversion program wanted to have only one conversion
and you did not know which? We have to be able to give the user a choice. In the
previous lab, you learned about the \pythoninline!input()! function that let you
``ask'' the user of your program a question. In this lab, you will learn how to
use \pythoninline!if!, \pythoninline!else!, and \pythoninline!elif! to have the program
choose one action out of multiple actions; for example, whether to convert to
Kelvin or to Fahrenheit.

Sometimes, you also want to be able to repeat a calculation for different
values. For example, you want to calculate the square root of all numbers
between 1 and 100. To do this, you do not have to actually repeat writing the
calculation in your code, there is the \pythoninline!while! statement to help you
repeat code.

\pagebreak

\tableofcontents

\pagebreak

\pagenumbering{arabic}
\section{On Types}

In Python, every value is associated with what we call a type. We have already
seen a few types in action: integers, floating-point numbers, strings, and
boolean values. The type of a value restricts the set of things that it can
represent.

An integer can be any whole number, for example \pythoninline{5}. A
float (floating-point number) is a number with decimal places, for example
\pythoninline{3.14} or \pythoninline{5.0}. A string is a sequence of characters
(letters, numbers, \ldots) enclosed by either double or single quotes, for
example \pythoninline{"I'm a string!"}. A boolean, as we just learned, can have
two values: \pythoninline{True} or \pythoninline{False}.

Most programming languages have types for a good reason: for one, operations
(such as \pythoninline{+}, \pythoninline{-}, \ldots) have different effects on
different types. For example, an integer \pythoninline{*} an integer results in
an integer (the multiplication of the two \emph{operands}), but an integer
\pythoninline{*} a string results in the string repeated. However, a string
\pythoninline{*} a string results in an error:
\begin{pyconcode}
>>> 5*3
15
>>> 5*'hi'
'hihihihihi'
>>> 'hi'*'hi'
Traceback (most recent call last):
  File "<stdin>", line 1, in <module>
TypeError: can't multiply sequence by non-int of type 'str'
\end{pyconcode}

In addition to that, some operations do not work on some types, which helps
ensure correctness of your program. Also, depending on the type of the operands,
Python has different implementations of what goes on under the hood, which can
then be optimized for each type separately.

Of course, there are ways to convert between types and there are ways to find
out what type the value of a specific variable is. To do so, use the function
\pythoninline{type()}.
\begin{pyconcode}
>>> type(5)
<class 'int'>
>>> x = 10
>>> type(x)
<class 'int'>
>>> x = 'Allons-y!'
>>> type(x)
<class 'str'>
>>> type(True)
<class 'bool'>
>>> x = "5.5" # string containing 5.5
>>> y = float("5.5") # convert string with 5.5 to a float
>>> y
5.5
>>> z = int("5") # integer containing 5
>>> z
5
>>> q = str(5) # string containing 5
>>> q
'5'
>>> # Note that an impossible conversion will throw an error
>>> int("55a")
Traceback (most recent call last):
  File "<stdin>", line 1, in <module>
ValueError: invalid literal for int() with base 10: '55a'
\end{pyconcode}

We call the process of converting a value from one type to another type
\emph{casting}. Note that only values have types, and variables can hold
different values.  Applying \pythoninline{type()} to a variable will give you
the type of the value that it holds at the time. In the previous section, we
learned about comparison operators such as \pythoninline{!=}, \pythoninline{<},
\pythoninline{>}, etc. What was not mentioned is that these comparison operators
also compare the type in some cases, but not in others:
\begin{pyconcode}
>>> 5.5 == '5.5' # comparing a string to a float does not work
False
>>> x = 5.5
>>> y = 5.5
>>> x == y
True
>>> 5.5 == float('5.5')
True
>>> 5.0 == 5 # comparing a float and an int works
True
\end{pyconcode}

The usual rule for what works and what does not work with comparisons is to
listen to your intuition: it makes sense to compare different types of numbers;
while it does not make sense to compare strings with numbers.

\subsection{Summary}

\begin{itemize}
  \item A type is a set of values that it can represent.
  \item Casting is the process of converting a value from one type to another.
  \item To find out what type a value / variable is, use the
    \pythoninline{type()} function.
  \item Types we know about at this point:
    \begin{multicols}{2}
      \begin{itemize}
        \item \pythoninline{str} -- string
        \item \pythoninline{int} -- integer number
        \item \pythoninline{float} -- floating-point number
        \item \pythoninline{bool} -- boolean value
      \end{itemize}
    \end{multicols}
    Not coincidentally, these type names are also the names of the casting
    functions.
\end{itemize}


\pagebreak
\section{Formatting Strings}

Previously, when we wanted to print out both a number and a string, we
had to resort to this:

\begin{pyconcode}
>>> x = 5
>>> print("x is equal to " + str(x))
x is equal to 5
\end{pyconcode}

However, there is an easier way to accomplish the same thing. By using the
\pythoninline!.format()! command, as shown below, we can have far more options for
how we format our output.

\begin{pyconcode}
>>> x = 5
>>> print("x is equal to {}".format(x))
x is equal to 5
\end{pyconcode}

Rather than leaving a gap in our string and then using \pythoninline{+} to add on
our variable, we instead include \pythoninline!{}! where we wish to place our
variable and add on \pythoninline{.format(x)} to the end of the string. This
replaces \pythoninline!{}! with the value of \pythoninline{x}.

If we include multiple instances of \pythoninline!{}! in our string, we can then
pass multiple variables to \pythoninline{.format()}. It will place them in the
string in the order provided.

\begin{pyconcode}
>>> x = 5
>>> y = 6
>>> print("x is equal to {} and y is equal to {}.".format(x, y))
x is equal to 5 and y is equal to 6.
\end{pyconcode}

We can also use \pythoninline!.format()! to control our output. For example, we
can restrict how many decimal places a floating point number is printed with. To
do this, we add \pythoninline{:.2f} inside of the \pythoninline!{}!. The
\pythoninline{.2f} specifies that we want 2 digits to follow the decimal point.
%If we wanted to, we
%could add an extra number before the colon to specify which of the arguments we
%want in this position. We don't want to mess with the order of the arguments, so
%we leave the position before the colon blank.

\begin{pyconcode}
>>> import math
>>> print(math.pi)
3.141592653589793
>>> print("{:.2f}".format(math.pi))
3.14
\end{pyconcode}

For more format options, see
\begin{center}
  \vspace{-2mm}
  \url{https://docs.python.org/3.1/library/string.html#format-string-syntax}
  \vspace{-2mm}
\end{center}

\subsection{Summary}

\begin{itemize}
  \item Syntax:
    \begin{python3code}
print("string containing {}".format(variable))
    \end{python3code}

    This will replace the \pythoninline!{}! with the value of
    \pythoninline{variable}.

  \item You can include multiple \pythoninline!{}! in a string and pass multiple
    values to \pythoninline{.format()}.

  \item You can specify advanced formatting options, such as number of digits
    after the decimal point.
\end{itemize}

\pagebreak
\section{Loops}
%Apply the above to while loops.

\subsection{While Loops}
The syntax of a \pythoninline{while} loop is very similar to that of an
\pythoninline{if} statement, but instead of only running the indented block of code
once, the \pythoninline{while} loop will continue running it until the given
boolean statement is no longer true.

\begin{python3code}
x = 10

while x > 0:
    print(x)
    x = x - 1
\end{python3code}

The above program will print out the numbers 10 to 1. Try stepping through this
program on paper, writing out the value of \pythoninline{x} at each time through
the loop. Then repeat for this modified version of the program:

\begin{python3code}
x = 10

while x > 0:
    x = x - 1
    print(x)
\end{python3code}

This version of the program will print out the numbers 9 to 0. This might seem a
bit strange, since the condition of the loop says it will stop when
\pythoninline{x} is no longer larger than 0. And yet, it prints out the value 0
before the loop ends. This is because the loop condition is only checked
whenever the end of the indented section is reached. If the condition is
\pythoninline{True}, then the indented section will be executed again. If the
condition is \pythoninline{False}, then the loop will end.

If the condition starts out \pythoninline{False}, then the loop will never execute.
The following program will not print anything:

\begin{python3code}
x = 0

while x > 0:
    x = x - 1
    print(x)
\end{python3code}

\pythoninline{if} and \pythoninline{else} can be combined with \pythoninline{while}, as
shown below:

\begin{python3code}
x = 10

while x > 0:
    if x % 2 == 0:
        print("{} is even.".format(x))
    else:
        print("{} is odd.".format(x))
    x = x - 1
\end{python3code}

Of course, they can be nested the other way around, too, with a
\pythoninline!while! inside conditional statements.

There can also be infinite while loops. Try the following:
\begin{python3code}
while True:
    print("Printing forever")
\end{python3code}

Press \emph{Ctrl+C} to stop the execution of this.

%\subsection{For Loops}
%
%Instead of using \pythoninline!while! loops, you can also use the \pythoninline!for!
%iterator (often also called \pythoninline!for! loop). The \pythoninline!for! loop
%allows you to ``iterate'' over a given list of things, for example a list of
%characters (a string):
%
%\begin{python3code}
%for c in "abc":
%    print("Hi, {}!".format(c))
%\end{python3code}
%
%Here, \pythoninline!c! is a variable you can use inside the code block of the
%for loop.
%
%The example will print:
%
%\begin{verbatimcode}
%Hi, a!
%Hi, b!
%Hi, c!
%\end{verbatimcode}
%
%A for loop can be over numbers as well, but this requires us to use the
%\pythoninline!range()! function. You can give \pythoninline!range()! a starting number
%and an end number:
%
%\begin{python3code}
%for num in range(0, 10):
%    print(num)
%\end{python3code}
%
%This will print the numbers from \emph{zero} to \emph{nine}. Notice that 10 is
%not included!
%
%You can also give \pythoninline!range()! a starting number, an end number, and an
%increment. The increment can be positive or negative. The numbers do not have to
%be actual numbers, you can give variables to it, too:
%
%\begin{python3code}
%start = 10
%end = 0
%increment = -2
%for number in range(start, end, increment):
%    print(number)
%\end{python3code}
%
%This will print:
%
%\begin{verbatimcode}
%10
%8
%6
%4
%2
%\end{verbatimcode}
%
%Notice how zero is not included. The end number is never included when you use
%\pythoninline!range()!.
%
%You can also just have the \pythoninline!for! loop ``iterate'' over a list of
%things:
%
%\begin{python3code}
%for number in [0, -2, 20, 24]:
%    print(number)
%\end{python3code}

\subsection{Nesting}

You can nest loops and conditional statements in any way you
like. The following is just an example:

\begin{python3code}
parity = input("Even or odd? ")

# prints even or odd numbers between 0 and 10, depending on user input
if parity == "odd":
    n = 1
    while n < 11:
        if n % 2 == 1:
            print(n)
        n = n + 2
elif parity == "even":
    n = 0
    while n <= 10:
        print(n)
        n = n + 2
else:
    print("You did not enter even or odd.")
\end{python3code}

\subsection{Summary}

\begin{itemize}
    \item Syntax:

      \begin{python3code}
while condition:
    # code to be repeated
      \end{python3code}

    This will repeat the indented code following the \pythoninline!while! until
    the condition is not true anymore. It checks the condition first, then runs
    the indented code, then checks the condition again, etc. Thus, if the
    condition is wrong in the first place, it will never run.

%  \item Syntax:
%
%    \begin{python3code}
%for variable in list:
%    # code doing something with variable
%    \end{python3code}
%
%    The list can be a string or a \pythoninline!range()! function or an actual list
%    delimited by brackets \pythoninline![]!. You will soon
%    learn that there is also a list data type in Python that you can use here
%    and other nice things, but we do not cover that in this lab.

  \item There can be infinite while loops.
  \item You can nest conditional statements and loops any way you want in any
    combination.
\end{itemize}

\pagebreak
\subsection{Exercises}
\label{subsec:whileex}

\begin{ex}[rps.py] Write a program that reads a character for playing the game of
    rock-paper-scissors. If the character entered by the user is not one of
    ``R'', ``P'', or ``S'', the program keeps on prompting the user to enter a
    character.

    % http://cs.smith.edu/dftwiki/index.php/CSC111_While_Loop_Exercises

    For example:

\begin{verbatimcode}
Enter R, P, or S >>> A
Did not enter R, P, or S. Try again.
Enter R, P, or S >>> R
You chose rock. Exiting.
\end{verbatimcode}
\end{ex}

\begin{ex}[sums.py] Write a program that keeps prompting the user for numbers to
  add to a sum until the user types in ``exit''. Then, display the sum of the
  numbers previously entered. Assume the user input is nothing other than
  numbers or ``exit''.

  For example:

  \begin{verbatimcode}
Enter a number to add to the sum >>> 15
Enter a number to add to the sum >>> 14.5
Enter a number to add to the sum >>> 12.25
Enter a number to add to the sum >>> exit
Sum of numbers: 41.75
  \end{verbatimcode}
\end{ex}

  \begin{ex}[primes.py] Write a program that checks if a number $N$ is prime.
    You have to ask the user for the number. Remember that a prime number is a
    number that is divisible only by $1$ and itself.

    A simple approach checks all numbers from $2$ up to $N$.

    Try to improve on the simple approach, though: do we really need to check
    all those numbers? At which point do you know that you can stop? Remember to
    \pythoninline!import math!.
  \end{ex}

\begin{ex}[fizzbuzz.py] Have the user enter a positive integer number. Then,
  print the numbers from 1 to that number each on a line. When the printed
  number is divisible by 3, print ``Fizz'', and when the number is divisible by
  5, print ``Buzz'', and when it is divisible by both, print ``FizzBuzz''.

  You must use \pythoninline!.format()! and a \pythoninline!while! loop.

  Should look like this when run:

  \begin{verbatimcode}
Enter a number: 16
1
2
3 Fizz
4
5 Buzz
6 Fizz
7
8
9 Fizz
10 Buzz
11
12 Fizz
13
14
15 FizzBuzz
16
  \end{verbatimcode}

  \begin{verbatimcode}
Enter a number: -1
Not a positive number!
  \end{verbatimcode}
\end{ex}

%  \begin{ex}[fizzbuzz\_for.py] Write the same fizzbuzz program using a
%    \pythoninline!for! loop and \pythoninline!range()! this time. Using
%    \pythoninline!.format()! is still required.
%  \end{ex}


%  \begin{ex}[dna.py] 
%    DNA is generally encoded with four letters: A, T, G, and C. For example, a
%    string of DNA would be ``ATTGCAT''.
%
%    You can find the ``complement'' of DNA for some biological reason (it is
%    double stranded, but that does not matter to us). The complement of A is T
%    and vice versa; the complement of G is C and vice versa.
%
%    Write a program that takes in a strand of DNA from the user using
%    \pythoninline!input()! and finds its complement. You may assume the user's
%    input is valid.
%
%    \begin{verbatimcode}
%Enter DNA: ATTGCAT
%Complement is: TAACGTA
%    \end{verbatimcode}
% \end{ex}



\pagebreak
\subsection{Exercises}
\label{subsec:turtleex}

  \begin{ex}[polygons2.py] Write a program that takes in a number using
    \pythoninline{input} and then draws a regular polygon with that many sides. A
    regular polygon is one where each side is the same length and each corner is
    the same angle. For example, the sample code given in the section draws a
    regular hexagon.

    Input:

    \begin{verbatimcode}
How many sides? 3
    \end{verbatimcode}

    Output:
    \begin{center}
%      \vspace{-10mm}
      \includegraphics[width=2.0in]{img/triangle}
%    \vspace{-10mm}
    \end{center}
\end{ex}

\begin{ex}[navigate.py] Write a program that takes directions from the command line
    to draw a line. Let the user input ``left'', ``right'', ``forward'', or
    ``stop''. Left and right turn the turtle left or right however many degrees
    are entered, forward moves the turtle forward (however far you wish), and
    stop ends the program. Please check the degrees for errors: they must be
    between 0 and 360 degrees! (Yes, Turtle could handle negative degrees, but
    we would like you to check.)

    Input:

  \begin{verbatimcode}
Please enter a direction: forward
Please enter a direction: left
How many degrees? 45
Please enter a directi/n: forward
Please enter a direction: left
How many degrees? -1
Invalid number, not moving.
Please enter a direction: left
How many degrees? 45
Please enter a direction: forward
Please enter a direction: forward
Please enter a direction: left
How many degrees? 45
Please enter a direction: left
How many degrees? 45
Please enter a direction: forward
Please enter a direction: right
How many degrees? 45
Please enter a direction: forward
Please enter a direction: stop
  \end{verbatimcode}

    Output:
    \begin{center}
%    \vspace{-5mm}
    \includegraphics[width=1.0in]{img/navigate}
%    \vspace{-10mm}
  \end{center}

\end{ex}

\pagebreak

\section{Submitting}

You should submit your code as a tarball. It should contain all files
used in the exercises for this lab. The submitted file should be named
\begin{center}
  \texttt{cse107\_firstname\_lastname\_lab2.tar.gz}
\end{center}

\begin{center}
  \textbf{Upload your tarball to Canvas.}
\end{center}

\listoftheorems

\end{document}
