% Project: Tanks
%
% CSE/IT 107: Introduction to Programming
% New Mexico Tech
%
% Prepared by Russell White and Christopher Koch
% Spring 2015
\documentclass[11pt]{cselabheader}
\usepackage{caption}

%%%%%%%%%%%%%%%%%% SET TITLES %%%%%%%%%%%%%%%%%%%%%%%%%
\fancyhead[R]{Project: Tanks}
\title{Project: Tanks}

\begin{document}

\maketitle

\hrule

\begin{quotation}
  ``When I see a bird that walks like a duck and swims like a duck and quacks like
  a duck, I call that bird a duck.''
\end{quotation}
\begin{flushright}
  --- James Whitcomb Riley, Duck Enthusiast
\end{flushright}

\begin{quotation}
``Imagination is more important than knowledge.''
\end{quotation}
\begin{flushright}
  --- Albert Einstein
\end{flushright}

\begin{quotation}
``The limits of my language mean the limits of my world.''
\end{quotation}
\begin{flushright}
  --- Ludwig Wittgenstein
\end{flushright}

\hrule


\section{Tanks}

\begin{warningbox}{Dirtbags Tanks}
  This project is inspired by Dirtbags Tanks by Neale Pickett
  More information is available at \url{http://dirtbags.net/tanks/}.
\end{warningbox}

You will be required to extend the tank class to three separate
``specialty'' tanks. In every case you will be overriding the
\pythoninline{ai} function to provide new behavior for your tank
type. Additionally, you are able to add sensors to your tank (to
better your AI experience). To understand this, see
\texttt{sample_tank.py}. In the \pythoninline{__init__} function of
\pythonineline{SampleTank}, two sensors are added to a sensor list. You, as
tankgineer

\begin{ex}[sample_tank.py]

\end{ex}


\pagebreak
\section{Submitting}
You should submit your code as a tarball. It should contain all files
used in the exercises for this lab. The submitted file should be named
\begin{center}
  \texttt{cse107\_firstname\_lastname\_tanks.tar.gz}
\end{center}

\begin{center}
  \textbf{Upload your tarball to Canvas.}
\end{center}

\listoftheorems

\end{document}

%%% Local Variables:
%%% mode: latex
%%% TeX-master: t
%%% End:
