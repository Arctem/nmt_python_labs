% Project: Tanks
%
% CSE/IT 107: Introduction to Programming
% New Mexico Tech
%
% Prepared by Russell White and Christopher Koch
% Spring 2015
\documentclass[11pt]{cselabheader}
\usepackage{caption}

%%%%%%%%%%%%%%%%%% SET TITLES %%%%%%%%%%%%%%%%%%%%%%%%%
\fancyhead[R]{Project: Tanks}
\title{Project: Tanks}

\begin{document}

\maketitle

\hrule

\begin{quotation}
  ``When I see a bird that walks like a duck and swims like a duck and quacks like
  a duck, I call that bird a duck.''
\end{quotation}
\begin{flushright}
  --- James Whitcomb Riley, Duck Enthusiast
\end{flushright}



\begin{quotation}
``Imagination is more important than knowledge.''
\end{quotation}
\begin{flushright}
  --- Albert Einstein
\end{flushright}

\begin{quotation}
``The limits of my language mean the limits of my world.''
\end{quotation}
\begin{flushright}
  --- Ludwig Wittgenstein
\end{flushright}

\hrule


\section{Tanks}

\begin{warningbox}{Dirtbags Tanks}
  This project is inspired by Dirtbags Tanks by Neale Pickett.
  More information is available at \url{http://woozle.org/tanks/intro.html}.
\end{warningbox}

\subsection{Project Overview}
You will be provided with several files comprising the Tank project. This is the
skeleton of a game involving several tanks. It consists of the files
\texttt{game.py}, \texttt{sample\_tank.py}, \texttt{sensor.py},
\texttt{tank.py}, and \texttt{tankutil.py}. For a description of what each of
these files do, stay here. For a description of your assignment, see
Section~\ref{subsec:ex}.

\subsubsection{game.py}
\texttt{game.py} contains two important things: \pythoninline{class Game} and
the \pythoninline{main()} function of the program. \pythoninline{Game}
contains the code that controls the interaction between tanks (collision
detection and firing) as well as the drawing of objects to the
\pythoninlin{tkinter} window. You should not have to worry about how it does
these things.

The \pythoninline{main()} function, however, you will need to edit, slightly.
This is where you will add your own tanks to the \pythoninline{Game} instance
that is already being created. To do this, simply imitate how the
\pythoninline{SampleTank}s are being added. Note that you will also need to add
an import at the start of the file in order to be able to access your tanks from
other files.

\subsubsection{tank.py}


\subsubsection{tankutil.py}


\subsubsection{sensor.py}


\subsubsection{sample\_tank.py}


\pagebreak
\subsection{Your Mission}
\label{subsec:ex}
You will be submitting 6 tanks that inherit from the \pythoninline{Tank} class
in \texttt{tank.py}. Each of your tanks should override, at the very least,
the \pythoninline{__init__} and \pythoninline{ai(self, delta)} methods of
the parent class.

\begin{ex}[coward.py]
%runs away from others
\end{ex}

\begin{ex}[charger.py]
%aggressively pursues anything it sees
\end{ex}

\begin{ex}[turret.py]
%can't move, aims at enemies
\end{ex}

\begin{ex}[elephant.py]
%bigger, slower, harder to kill
\end{ex}

\begin{ex}[mouse.py]
%smaller and faster
\end{ex}

\begin{ex}[custom.py]
%your choice
\end{ex}


\subsection{Bug Bounty}


\section{Submitting}
You should submit your code as a tarball. It should contain all files
used in the exercises for this lab. The submitted file should be named
\begin{center}
  \texttt{cse107\_firstname\_lastname\_tanks.tar.gz}
\end{center}

\begin{center}
  \textbf{Upload your tarball to Canvas.}
\end{center}

\listoftheorems

\end{document}
